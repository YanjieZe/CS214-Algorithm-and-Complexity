\documentclass[12pt,a4paper]{article}
\usepackage{amsmath,amscd,amsbsy,amssymb,latexsym,url,bm,amsthm}
\usepackage{epsfig,graphicx,subfigure}
\usepackage{enumitem,balance}
\usepackage{wrapfig}
\usepackage{mathrsfs,euscript}
\usepackage[usenames]{xcolor}
\usepackage{hyperref}
\usepackage{booktabs}
\usepackage[vlined,ruled,linesnumbered]{algorithm2e}
\usepackage{threeparttable}
\hypersetup{colorlinks=true,linkcolor=black}

\newtheorem{theorem}{Theorem}
\newtheorem{lemma}[theorem]{Lemma}
\newtheorem{proposition}[theorem]{Proposition}
\newtheorem{corollary}[theorem]{Corollary}
\newtheorem{exercise}{Exercise}
\newtheorem*{solution}{Solution}
\newtheorem{definition}{Definition}
\theoremstyle{definition}

\renewcommand{\thefootnote}{\fnsymbol{footnote}}

\newcommand{\postscript}[2]
 {\setlength{\epsfxsize}{#2\hsize}
  \centerline{\epsfbox{#1}}}

\renewcommand{\baselinestretch}{1.0}

\setlength{\oddsidemargin}{-0.365in}
\setlength{\evensidemargin}{-0.365in}
\setlength{\topmargin}{-0.3in}
\setlength{\headheight}{0in}
\setlength{\headsep}{0in}
\setlength{\textheight}{10.1in}
\setlength{\textwidth}{7in}
\makeatletter \renewenvironment{proof}[1][Proof] {\par\pushQED{\qed}\normalfont\topsep6\p@\@plus6\p@\relax\trivlist\item[\hskip\labelsep\bfseries#1\@addpunct{.}]\ignorespaces}{\popQED\endtrivlist\@endpefalse} \makeatother
\makeatletter
\renewenvironment{solution}[1][Solution] {\par\pushQED{\qed}\normalfont\topsep6\p@\@plus6\p@\relax\trivlist\item[\hskip\labelsep\bfseries#1\@addpunct{.}]\ignorespaces}{\popQED\endtrivlist\@endpefalse} \makeatother

\begin{document}
\noindent

%========================================================================
\noindent\framebox[\linewidth]{\shortstack[c]{
\Large{\textbf{Lab01-Algorithm Analysis}}\vspace{1mm}\\
CS214-Algorithm and Complexity, Xiaofeng Gao, Spring 2021.}}
\begin{center}
\footnotesize{\color{red}$*$ If there is any problem, please contact TA Haolin Zhou. Also please use English in homework.}

% Please write down your name, student id and email.
\footnotesize{\color{blue}$*$ Name:Yanjie Ze  \quad Student ID: 519021910706\quad Email: zeyanjie@sjtu.edu.cn}
\end{center}

\begin{enumerate}


\item \textit{Complexity Analysis.} Please analyze the time and space complexity of Alg.~\ref{Alg-quicksort} and Alg.~\ref{Alg-cocktailsort}. \par

\begin{minipage}[t]{0.45\textwidth}
	\begin{algorithm}[H]
		\KwIn{An array $A[1,\cdots,n]$}
		\KwOut{$A[1,\cdots,n]$ sorted nondecreasingly}
		
		\BlankLine
		\caption{QuickSort}\label{Alg-quicksort}
		
		%\If{$n \le 1$}{
		%  \Return\;
		%}
		
		$pivot \leftarrow A[n]$; $i \leftarrow 1$\;
		\For{$j \leftarrow 1$ \KwTo $n-1$}{
			\If{$A[j] < pivot$}{
				swap $A[i]$ and $A[j]$\;
				$i \leftarrow i+1$\;
			}
		}
		
		swap $A[i]$ and $A[n]$\;
		\lIf{$i>1$}{$\operatorname{QuickSort}(A[1,\cdots,i-1])$}
		\lIf{$i<n$}{$\operatorname{QuickSort}(A[i+1,\cdots,n])$}
	\end{algorithm}
\end{minipage}
\hfill
\begin{minipage}[t]{0.45\textwidth}
\begin{algorithm}[H]
\KwIn{An array $A[1,\cdots,n]$}
\KwOut{$A[1,\cdots,n]$ sorted nonincreasingly}
\BlankLine
\caption{CocktailSort}
\label{Alg-cocktailsort}
\BlankLine
	$i\leftarrow 1;$ $j\leftarrow n;$$sorted\leftarrow false$\;
	\While{\textbf{not} sorted}{
		$sorted \leftarrow true$\;
		\For{$k\leftarrow i$ \textbf{to} $j-1$}{
			\If{$A[k] < A[k+1]$}{
				swap $A[k]$ and $A[k+1]$\;
				$sorted\leftarrow false$\;
			}
		}
		$j\leftarrow j-1$\;
		

		\For{$k\leftarrow j$ \textbf{downto} $i+1$}{
			\If{$A[k-1] < A[k]$}{
				swap $A[k-1]$ and $A[k]$\;
				$sorted\leftarrow false$\;
			}
		}
		$i\leftarrow i+1$\;
	}
\end{algorithm}
\end{minipage}

\begin{enumerate}
	 
\item Fill in the blanks and \textbf{explain} your answers. You need to answer when the best case and the worst case happen. 
\begin{table}[!h]

\label{Tab-compare}
	\centering
	\begin{threeparttable}
	\begin{tabular}{c|c| c }
		\toprule[2pt]
		\textbf{Algorithm} & \textbf{Time Complexity}\tnote{1} & \textbf{Space Complexity} \\
		\hline
		\hline
		$QuickSort$ &  &  \\

		$CocktailSort$ &  &   \\
		\bottomrule[2pt]


	\end{tabular}
    \begin{tablenotes}
    	\footnotesize
    	\item[1] The response order can be given in \emph{best}, \emph{average}, and \emph{worst}.
    \end{tablenotes}
	\end{threeparttable}
\end{table}

\item For Alg.~\ref{Alg-quicksort}, how to modify the algorithm to achieve the same expected performance as the \textbf{average} case when the \textbf{worst} case happens?
\end{enumerate} 
%    \begin{solution}
%       Uncomment this block to write your proof.
%    \end{solution}

\item \textit{Growth Analysis.} Rank the following functions by order of growth with brief explanations: that is, find an arrangement $g_1, g_2, \ldots , g_{15}$ of the functions $g_1 = \Omega(g_2), g_2 = \Omega(g_3), \ldots, g_{14} = \Omega(g_{15})$.  Partition your list into equivalence classes such that functions $f(n)$ and $g(n)$ are in the same class if and only if $f(n) = \Theta(g(n))$. Use symbols ``$=$'' and ``$\prec$'' to order these functions appropriately. Here $\log n$ stands for $\ln n$.
$$
\begin{array}{ccccc}
	1 \quad & \quad n \quad & \quad \log n \quad & \quad \log (\log n) \quad & \quad n \log n \\
	\log_4 n \quad & \quad 2^n \quad & \quad 4^n \quad & \quad 2^{\log n} \quad & \quad 2^{2^n} \\
	\log (n!) \quad & \quad n! \quad & \quad (2n)! \quad & \quad  n^{1/2} \quad & \quad n^2 \\
\end{array}
$$
%    \begin{solution}
%       Uncomment this block to write your proof.
%    \end{solution}


\end{enumerate}

\vspace{20pt}

\textbf{Remark:} You need to include your .pdf and .tex files in your uploaded .rar or .zip file.

%========================================================================
\end{document}
